\documentclass[../SRP.tex]{subfiles}

\begin{document}

\begin{titlepage}
  % Defines a new command for horizontal lines, change thickness here
  \newcommand{\HRule}{\rule{\linewidth}{0.5mm}}

  % Centre everything on the page
	\center

  % Main heading such as the name of your university/college
	\textsc{\LARGE Aarhus Tech} \\[0.25cm]
  \textsc{\Large SRP} \\[0.5cm]

  % Major heading such as course name
	%\textsc{\Large Major Heading}\\[0.5cm]

  % Minor heading such as course title
	%\textsc{\large Minor Heading}\\[0.5cm]

  {\large\today}
  \\[1cm]

	\HRule\\[0.4cm]
  % Title of your document
	{\huge\bfseries Monte Carlo Lokalisering}\\[0.1cm]

	\HRule\\[1.5cm]

  % Author(s)
	\begin{minipage}{0.4\textwidth}
		\begin{flushleft}
			\large
			\textit{Forfatter}\\
			Jacob Emil Ulvedal Rosborg % Your name
		\end{flushleft}
	\end{minipage}
	~
	\begin{minipage}{0.4\textwidth}
		\begin{flushright}
			\large
			\textit{Vejleder}\\
			Mikkel Stouby Petersen\\
			Jørn Sanggaard
		\end{flushright}
	\end{minipage}\\[4.0cm]

  \begin{center}
  \begin{minipage}{0.8\textwidth}
    \center{\Large{\textbf{Abstract}}}\\
    \justify
    This paper is concerned with the topic of the Monte Carlo Method and will explain how it is perceived to be a general thought of mind that have evolved into different methods and algorithms of approximating problems. It was first developed by Stanislaw Ulam during the Manhattan Project and was used to solve the problem of neutron diffusion. The problem of neutron diffusion is complex, so much in fact that they could not derive it algebraically. They therefore looked into means of approximating neutrons diffusion which let to the development of the Monte Carlo Method. Furthermore, this paper will explain how the method was adopted to solve integrates that are not otherwise solvable and how random sampling is the key to doing so. The methods for solving such integrates are often referred to as numerical integration and in the paper we will be using the method Hit-or-Miss to approximate $\pi$. It is illustrated how the method can be used to locate an autonomous robot in a given localized area. Finally, it is discussed how the method of localizing is not inherently robust, but can be made so by introducing noise particles. 
  \end{minipage}
  \end{center}
\end{titlepage}

\end{document}
