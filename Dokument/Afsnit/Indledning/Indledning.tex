\documentclass[../../SRP.tex]{subfiles}

\begin{document}

\chapter{Indledning}

Nogle problemer er så komplicerede at de næsten er umulige at analysere, modellere og løse algebraisk. For eksempel den kaotiske og tilfældige proces, der finder sted når man spalter uran-235 (neutron diffusion). Dette problem stod forskerne overfor i 1940 under udviklingen af atombomben i Manhattan projektet. Her blev Monte Carlo Simulering anvendt til at simulere neutroners vandring, og dette blev brugt til at vurdere de optimale fysiske forhold for den kædereaktion, der skulle få bomben til at sprænge \cite{AHF}. \\

Begrebet Monte Carlo Metod\textit{en} dækker egentlig over en række metoder, der kan bruges til at analyse problemer som ikke fremstår løsbare, såsom \textit{neutron diffusion}. For eksempel findes der metoder, som Monte Carlo Simulering og Monte Carlo Lokalisering og ud over disse er der yderligere metoder inden for Finans og Medicin. Denne opgave vil fokusere på numerisk integration og lokalisering. Jeg vil yderligere komme ind på Stokastiske Variabler og deres egenskaber, samt anvendelse og implementering af Monte Carlo Lokalisering hvilket er en form for partikelfilter i en autonom robot. \\

Monte Carlo Metoden blev først rigtigt anvendt da computere blev så udviklede at de kunne udføre disse simuleringer, også kaldet eksperimenter, for os. Da disse simuleringer kræver et forholdsvis stort antal gentagelser, et antal der både ville have krævet arbejdskraft og tid af umådelige proportioner. Alt dette for at opnå et resultat der ikke altid ville være brugbart. Dette resultat vil nemlig altid være en approximering hvorimod en algebraisk tilgang ville udlede værdien og derved være eksakt. Til gengæld er man med metoden i stand til at tackle problemer der så komplekse i sin natur at det ikke er praktisk muligt at udlede disse problemer algebraisk \cite{SBM} \\

\section{Opgaveformulering}
  \begin{enumerate}[label=(\Roman*)]
  \item Redegør for, hvad Monte Carlo-algoritmer er, og giv eksempler både praktiske og teoretiske anvendelser. For eksempel i forbindelse med numerisk integration.

  \item Forklar centrale egenskaber ved stokastiske variable i det omfang det er nødvendigt for at forstå algoritmernes virkemåde.

  \item Vis, hvordan Monte Carlo-algoritmen kan anvendes til lokalisering af robotter. Kom herunder ind på, hvordan algoritmen kan implementeres.

  \item Diskuter Monte Carlo-metodens muligheder og begrænsninger i forbindelse med anvendelse i en konkret autonom robot.
  \end{enumerate}

\section{Afgrænsning}
Denne opgave henvender sig til studerende på 3. årgang på en gymnasial uddannelse. For at læse opgaven kræves der ikke en dybdegående forståelse for hverken statistik eller algoritmer. Opgaven vil forklare de begreber der er nødvendige for at forstå Monte Carlo Metoden, men kun på et redegørende niveau. 

\end{document}
