\documentclass[../../SRP.tex]{subfiles}

\begin{document}

\chapter{Indledning}

Nogle problemer er så komplicerede, at de næsten er umulige at analysere, modellere og løse algebraisk. For eksempel den kaotiske og tilfældige proces, der finder sted når man spalter uran-235 (neutron diffusion). Dette problem stod forskerne overfor i 1940erne under udviklingen af atombomben i Manhattan projektet. Her blev Monte Carlo Simulering anvendt til at simulere neutroners vandring. Det blev brugt til at vurdere hvor langt neutroner fra diverse henfald ville bevæge sig i gennem forskellige materialer. Denne viden blev brugt til at designe reaktorer med tilstrækkelig beskyttelse \cite{AHF}. \\

Begrebet Monte Carlo Metod\textit{en} dækker over en række metoder, der kan bruges til at analyse problemer, som ikke fremstår løsbare, såsom \textit{neutron diffusion}. For eksempel findes der metoder, som Monte Carlo Simulering og Monte Carlo Lokalisering. Variationer af disse metoder findes også indenfor Finans og Medicin. 

Denne opgave vil fokusere på numerisk integration ved brug af Monte Carlo Metoden Hit-or-Miss og ydermere vil opgaven komme ind på lokalisering af en autonom robot ved brug af Monte Carlo Lokalisering. Derudover vil opgaven også redegøre for Stokastiske Variabler, og de egenskaber der er nødvendige for at forklare Monte Carlo metoderne. \\

Monte Carlo Metoden blev først rigtigt anvendt da computeren blev så teknologisk udviklet, at den kunne udføre disse simuleringer, også kaldet eksperimenter, for os. Disse simuleringer kræver et forholdsvis stort antal gentagelser, et antal der både kræver arbejdskraft og tid af umådelige proportioner. Alt dette for at opnå et resultat, der ikke altid vil være brugbart. Dette resultat vil være en approximering hvorimod en algebraisk tilgang vil udlede værdien og derved være eksakt. Til gengæld er man med Monte Carlo Metoden i stand til at tackle problemer der så komplekse i sin natur, at det ikke er praktisk muligt at udlede disse problemer algebraisk \cite{SBM}. \\

I nedenstående fremgår denne SRP opgaves opgaveformulering. Det er netop denne, som vil blive besvaret løbende gennem opgaven. SRP opgavens skrives med udgangspunkt i fagene Matematik A og Robotteknik A. \\

\section{Opgaveformulering}
  \begin{enumerate}[label=(\Roman*)]
  \item Redegør for, hvad Monte Carlo-algoritmer er, og giv eksempler både praktiske og teoretiske anvendelser. For eksempel i forbindelse med numerisk integration.

  \item Forklar centrale egenskaber ved stokastiske variable i det omfang det er nødvendigt for at forstå algoritmernes virkemåde.

  \item Vis, hvordan Monte Carlo-algoritmen kan anvendes til lokalisering af robotter. Kom herunder ind på, hvordan algoritmen kan implementeres.

  \item Diskuter Monte Carlo-metodens muligheder og begrænsninger i forbindelse med anvendelse i en konkret autonom robot.
  \end{enumerate}



\end{document}
