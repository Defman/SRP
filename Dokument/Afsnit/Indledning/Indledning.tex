\documentclass[../../SRP.tex]{subfiles}

\begin{document}

\chapter{Indledning}
Nogle problemer er så kompliceret at de næsten er umulige at analysere, modeller og løse algebraisk, f.eks den kaotiske og tilfældige process der finder sted når man spalter uran-235 (neutron diffusion).
 Disse problemer kan dog løses ved brug af Monte Carlo Simulering, dette kan lade sig gøre ved at lave en statistisk model af de hændelser der må finde sted. Der efter ved brug af computer kraft simuleres de forskelige tilfælde og der findes frem til sandsyneligheden for de forskellige udkom.

Monte Carlo Metoder er en bred klasse af statistiske metoder der afhænger af gentagen sandsyneligheden af stokastiske variabler, til

Monte Carlo Simulering er en metode der hører under Monte Carlo Metoder

Jeg vil komme ind på hvordan Monte Carlo Methoden kan bruges til at finde integralet af funktioner der ikke er differentiable, f.eks. funktioner der ikke er kontinuert.

Monte Carlo Lokalisering er en form for partikel filter der kan bruges til at lokalisere et objekt på et kendt kort ved brug 


\section{Opgaveformulering}
  \begin{enumerate}[label=(\Roman*)]
  \item Redegør for, hvad Monte Carlo-algoritmer er, og giv eksempler både praktiske og teoretiske anvendelser. For eksempel i forbindelse med numerisk integration.

  \item Forklar centrale egenskaber ved stokastiske variable i det omfang det er nødvendigt for at forstå algoritmernes virkemåde.

  \item Vis, hvordan Monte Carlo-algoritmer kan anvendes til lokalisering af robotter. Kom herunder ind på, hvordan algoritmen kan implementeres.

  \item Diskuter Monte Carlo-metodens muligheder og begrænsninger i forbindelse med anvendelse i en konkret autonom robot.
  \end{enumerate}

\section{Afgrænsning}


\end{document}
