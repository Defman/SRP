\documentclass[../../SRP.tex]{subfiles}
  
\begin{document}

\chapter{Diskussion}

En begrænsning af Monte Carlo Metoden er, at den ikke kan bruges til at lokalisere en robot i omgivelser der er dynamiske. Dette skal forstås på den måde, at hvis robottens interne kort ikke længere stemmer overens med hvad sensorne måler vil robotten ikke kunne tilnærme sig sin lokalisering. Dette vil altså betyde at, hvis robotten måler afstanden til vægge vil den for eksempel blive forvirret når mennesker bevæger sig omkring den. Den vil her forveksle menneskerne med statiske murer, og tro at den er meget tættere på en mur end den i virkligheden er \cite{ML}. \\

En anden begrænsning ved Monte Carlo Lokalisering er Robot-kidnapnings problemet.
Monte Carlo Lokalisering i forhold til dennes implementering kan både være robust og ikke robust over for dette problem. Robot kidnapnings problemet består i at robotten for eksempelvis bliver samlet op og flyttet til et nyt punkt uden at blive informeret omkring dette \cite{ML}. Jeg vil mene, at dette problem kan løses ved at introducere interferens i form af et antal partikler, som er symmetrisk fordelt hen over udfaldsrummet. Dette vil sandsynligvis have den effekt, at hvis robotten blev flyttet til et nyt punkt, ville interferens give mulighed for at de kommende iterationer kunne danne en koncentration omkring disse partikler, som ikke er i nærheden af den før hen tilnærmede lokalisering. Altså vil interferens fungere som om en form for kickstart. \\

Et tredje eksempel på en begrænsning eller en mulighed i forbindelse med Monte Carlo lokalisering er robotter der bevæger sig i rummet. Hvis vi betragter en robot, der skal lokalisere sig selv i rummet i stedet for i planet skal der bruges et større antal partikler. Dette vil betyde, at hvis man ønsker at nedbringe fejlen af aproximering med en faktor $10$ skal man øge antallet af partikler med $10^3$ i modsætning til planet, hvor man skal bruge $10^2$ partikler. At simulere denne mængde partikler kan være en begrænsning for de små microprocessor, der driver vores robotter, men i takt med at vi udvikler mindre og hurtigere processorer vil det blive muligt at øge antallet af partikler der simuleres samtidigt \cite{ML}. \\

Desuden er Monte Carlo Lokaliserings præcision afhængig af, hvor præcist vi kan bestemme fejlen mellem vores interne kort, og derved partiklerne, og de målinger vores sensorer registrerer. Fejl her kunne eksempelvis skyldes, at en afstandssensor har en nøjagtighed på $\pm1cm$. Med et stort nok antal partikler ville dette problem forsvinde da vi udvælger en håndfuld partikler, der har den lavest fejl, og derved overlever kun de partikler med de mindste fejl. For tilfældet med en farve sensor kan fejlen skyldes at målingerne kan variere med eksempelvis op til $10\%$.

\end{document}
