\documentclass[../../SRP.tex]{subfiles}
  
\begin{document}

\chapter{Diskussion}

En begrænsning af Monte Carlo Metoden er, at den ikke kan bruges til at lokalisere en robot i omgivelser der er dynamiske. Dette skal forstås på den måde at hvis robottens interne kort ikke længere stemmer overens med hvad sensorne måler vil robotten ikke kune tilnærme sig sin lokalisering. Dette vil altså betyde at, hvis robotten måler afstanden til vægge vil den f.eks. blive forvirret når mennesker bevæger sig omkring den. Da den ville forveksle menneskerne med statiske murer, og tro at den er meget tættere på en mur end den i virkligheden er. \\

Monte Carlo Lokalisering i forhold til dennes implementering kan både være robust og ikke robust over for Robot kidnapnings problemet. Robot kidnapnings problemet involverer at robotten for eksempelvis bliver samlet op og flyttet til et nyt punkt uden at blive fortalt omkring dette \cite{ML}. Jeg vil mene at dette problem kan løses ved at introducere interferens i form af et antal partikler som er symmetrisk fordelt hen over udfaldsrummet. Dette vil have den effekt at hvis robotten blev flyttet til et nyt punkt, ville interferens give mulighed for at robotten kunne finde sig på en af de nye partikler som ikke er i nærheden af den tilnærmede lokalisering. Altså interferens ville fungere som om en form for kickstart. \\

Hvis vi betragter en robot der skal lokalisere sig selv i rummet i stedet for planet skal der bruges et større antal partikler. Dette vil betyde at hvis man ønsker at nedbringe fejlen med en faktor $10$ skal man øge antallet af partikler med $10^3$ i modsætning til planet hvor man skal bruge $10^2$ partikler. At simulere denne mængde partikler er en begrænsning for de små microprocessor der driver vores robotter, men i takt med at vi udvikler mindre og hurtigere processorer vil det blive muligt at øge antallet af partikler der simuleres samtidigt \cite{ML}. \\

Altså Monte Carlo Lokaliserings præcision afhænger af hvor præcist vi kan bestemme fejlen mellem vores interne kort og derved partikler og de målinger vores sensorer registrerer. Fejl her kunne f.eks. skyldes at en afstands sensor har en nøjagtighed på $\pm1cm$. Med et stort nok antal partikler ville dette problem forsvinde da vi udvælger en håndfuld partikler der har den lavest fejl, og derved overlever kun de partikler med de mindste fejl.

\end{document}
