\documentclass[../../SRP.tex]{subfiles}

\begin{document}

\chapter{Konklusion}

Monte Carlo Metoden blev udviklet under anden verdenskrig, og siden da er metoden blevet adopteret til at kunne løse forskellige problemer, der er komplekse eller tilfældige i deres natur. \\

I denne opgave er der blevet redegjort for, hvordan Monte Carlo Metoden kan anvendes i praksis og teori samt hvorledes en autonom robot kan implementere en sådan algoritme til at lokalisere sig selv i kendte omgivelser. Mulighederne for Monte Carlo Metoden er mange, men grundet af at antallet af simuleringer der skal til for at reducere fejlen, er i mange tilfælde så hurtigt voksende at man bliver begrænset af vores nuværende regne kraft. \\

I afsnittet om stokastiske variabler blev der fundet frem til en række egenskaber, heriblandt den forventede værdi. Disse egenskaber kunne bruges til at beskrive Monte Carlo Metoden og samt udførelsen af en række eksperimenter og simuleringer kan udføres. \\

Monte Carlo Metoden kan benyttes inden for mange områder, blandt andet kan metoden Hit-or-Miss bruges til at integrere en figur eller funktion, der ikke kan udledes algebraisk. I opgaven blev metoden anvendt til at integrere enhedscirklen og derved udlede $\pi$. \\

\end{document}
