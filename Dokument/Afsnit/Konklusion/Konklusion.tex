\documentclass[../../SRP.tex]{subfiles}

\begin{document}

\chapter{Konklusion}
Monte Carlo Metoden blev udviklede under anden verdens krig, og siden da blevet adopteret til at kunne løse forskellige problemer der er komplekse eller tilfældige i deres natur. I denne opgave er der blevet redegjort for hvordan Monte Carlo Metoden kan anvendes i pratisk og dens teori samt hvorledes en autonom robot kunne implementere et sådan algoritme til at lokalisere sig selv i kendte omgivelser. Mulighederne for Monte Carlo Metoden er mange, men på grund af at antallet af simulering der skal til for at reducere fejlen ikker er proportional med reduktionen, altså for at reducere fejlen med $10$ skal antallet af simuleringer i nogle tilfælde øges med en faktor $100$.

Monte Carlo Metoden kan benyttes inden for mange områder, heriblandt kan metoden Hit-or-Miss bruges til at integrere en figur der ikke kan udledes algebraisk, vi anvendte metoden til at integrere enhedscirklen og derved udlede $\pi$.

Monte Carlo Metoden kan også anvendes til at lokalisere en autonom robot, og kan gøre dette forholdsvist robust.

\end{document}
