\documentclass[../../SRP.tex]{subfiles}

\begin{document}

\chapter{Konklusion}
Monte Carlo Metoden blev udviklet under anden verdens krig, og siden da er metoden blevet adopteret til at kunne løse forskellige problemer, der er komplekse eller tilfældige i deres natur.  \\

I denne opgave er der blevet redegjort for, hvordan Monte Carlo Metoden kan anvendes i praksis og teoretisk samt hvorledes en autonom robot kan implementere en sådan algoritme til at lokalisere sig selv i kendte omgivelser. Mulighederne for Monte Carlo Metoden er mange, men på grund af at antallet af simuleringer, der skal til for at reducere fejlen, da det ikke er proportional med reduktionen af fejlen. Dette betyder, at for at reducere fejlen med en faktor $10$ skal antallet af simuleringer i nogle tilfælde øges med en faktor $100$. \\

I afsnittet om stokastiske variabler blev der fundet frem til en række egenskaber, her i blandt den forventede værdi. Dette kunne bruges til at beskrive Monte Carlo Metoden og hvorledes en række eksperimenter og simuleringer kan udføres. \\

Monte Carlo Metoden kan benyttes inden for mange områder, heriblandt kan metoden Hit-or-Miss bruges til at integrere en figur eller funktion, der ikke kan udledes algebraisk. I opgaven blev metoden anvendt til at integrere enhedscirklen og derved udlede $\pi$. \\

\end{document}
