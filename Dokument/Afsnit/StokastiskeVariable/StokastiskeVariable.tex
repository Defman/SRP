\documentclass[../../SRP.tex]{subfiles}

\begin{document}

\chapter{Stokastiske Variable}

For at forstå Stokastiske Variabler kræves der en forståelse for nogle simplere begreber inden for statistiske. Disse begreber er udfaldsrum, delmængde, sandsynlighedsfunktion samt sandsynlighedsfelt. Disse begreber er essentielle inden for statistisk.

\section{Udfaldsrum og delmængder}

Der findes to type af udfaldsrum, henholdsvis et diskret og kontinuert udfaldsrum. Forskellen på disse to er at der i et diskret udfaldsrum kun er et givet antal mulige udfald. Et eksempel på et diskret udfaldsrum kunne være en sekssidet terning. Denne sekssidet terning vil have et udfaldsrum $U$ der indeholder elementerne $1,2,3,4,5,6,7$ dette kan også skrives som $U = \{1,2,3,4,5,6\}$. Altså vi har tale om et udfaldsrum med længden seks, hvilket skrives $|U| = 6$. Hvorimd et kontinuert udfaldsrum kunne være højden på elever i en klasse, udfaldsrummet kunne være defineret som $U = [150cm;210cm]$. Udfaldsrummet er kontinuert da elevers højde kan variere med et infinitesimal, altså elev $A$ kunne være meget meget lidt højere end elev $B$, altså en uendelig lille mængde højere. I sådan et udfaldsrum findes der ikke nogen længde af udfaldsrummet, altså $|U| \notin \mathbb{R}$ \cite{SC}. \\

En delmængde er en mængde af udfaldsrummet, det beskrives ofte med $A$ eller $B$. Hvis vi tager udgangspunkt i den sekssidet terning igen, kunne en delmængde af dets udfaldsrum være $A = \{1,3\}$. Altså en delmængde indeholder altså en eller flere elementer fra udfaldsrummet. \\

Yder mere vil et bestemt element i et udfaldsrum eller delmængde denoteres $u_i$ eller $a_i$, altså for henholdsvis $A$ og $U$. Det vil sige hvis vi har en mængde $Q$ vil et bestemt element i $Q$ denoteres som $q_i$. Desuden findes der en række notationer for der beskriver delmængders relationer til hinanden og udfaldsrummet.
\begin{enumerate}
  \item $A \cup B$ : udfaldet ligger i enten A eller B, evt. i både A og B
  \item $A \cap B$ : udfaldet ligger i både A og B
  \item $A \ B$ : udfaldet ligger i enten A eller B, evt. i både A og B
  \item $A \cup B$ : udfaldet ligger i enten A eller B, evt. i både A og B
\end{enumerate}

\section{Sandsynlighedsfunktion}

Sandsynlighedsfunktion denoteres $P$ og beskriver sandsynligheden for et element i et udfaldsrummet $U$. For eksempel kunne $P(\{1,2\}) = 0.8$ og $P(3) = 0.2$ hvor udfaldsrummet $U = \{1,2,3\}$. Dette betyder at sandsynligheden for $1$ eller $2$ er $80\%$ hvor i mod sandsynligheden for $3$ er $20\%$. Desuden kan vi bruge additions loven til at udlede den totale sandsynlighed for udfaldsrummet \cite{SC}.
\begin{align}
  P(1) + P(2) + P(3) &= 1 \\
  P(\{1,2\}) + P(3) &= 1
\end{align}
Desuden kan $\{1,2\}$ også beskrives som en delmængde af U på formen $A = \{1,2\}$ af $U$ altså så det er gældende at
\begin{equation}
  P(A) = P(1) + P(2)
\end{equation}

\section{Sandsynlighedsfelt}

Et sandsynlighedsfelt findes på to former, vi vil kun beskæftige os med det endelige sandsynlighedsfelt. Det endelig sandsynlighedsfelt $(U, P)$ er bestående af et udfaldsrum $U$ og en sandsynlighedsfunktion $P$. Hvis der er tale om et symetrisk sandsynlighedsfelt, vil følgende være gældende for $P$.
\begin{align}
  P(U) &= 1 \\
  P(U_1) + P(U_i) + ... + P(U_n) &= 1
\end{align}
De to udsagn udtrykker derved at sandsynligheden for $P(U_i)$ er i intervallet $[0;1]$ samt at den samlede sandsynlighed for feltet er $1$, hvor $1 = 100\%$. 

 et endeligt og et  er bestående af både et udfaldsrum og dens delmængder samt en sandsynligheds funktion der beskrive sandsynligheden for hvert given  \cite{SC}

\section{Stokastiske variabler}

Stokastiske variabler findes af to type der begge beskrives med $X$, disse to type er henholdsvis diskret og kontinuert. Diskret stokastiske variabler har et endeligt udfaldsrum,

Stokastiske variabler opfører sig på en måde hvor $X$ variables tilstand er alle mulige udfald i udfaldsrummet $U$ i sandsynlighedsfeltet med sandsynlighederne $P(U_n)$

\end{document}
