\documentclass[../../SRP.tex]{subfiles}

\begin{document}

\chapter{Stokastiske Variable}

For at forstå Stokastiske Variabler kræves der en forståelse for nogle simplere begreber inden for statistik. Disse begreber er udfaldsrum, delmængde, sandsynlighedsfunktion samt sandsynlighedsfelt. De nævnte begreber er essentielle inden for statistik.

\section{Udfaldsrum og delmængder}

Der findes to typer af udfaldsrum, henholdsvis et diskret og kontinuert udfaldsrum. Forskellen på disse to er at der i et diskret udfaldsrum kun er et givent antal mulige udfald. Et eksempel på et diskret udfaldsrum kunne være en sekssidet terning. Denne sekssidet terning vil have et udfaldsrum $U$ der indeholder elementerne $1,2,3,4,5,6,7$ dette kan også skrives som $U = \{1,2,3,4,5,6\}$. Altså vi har tale om et udfaldsrum med længden seks, hvilket skrives $|U| = 6$. Hvorimod et kontinuert udfaldsrum kunne være højden på elever i en klasse, udfaldsrummet kunne være defineret som $U = [150cm;210cm]$. Udfaldsrummet er kontinuert da elevers højde kan variere med et infinitesimal, altså elev $A$ kunne være en uendeligt lille mængde højere end elev $B$. I sådan et udfaldsrum findes der ikke nogen længde af udfaldsrummet, altså $|U| \notin \mathbb{R}$ \cite{SC}. \\

En delmængde er en mængde af udfaldsrummet, det beskrives ofte med $A$ eller $B$. Hvis vi tager udgangspunkt i den sekssidet terning igen, kunne en delmængde af dets udfaldsrum være $A = \{1,3\}$. Altså en delmængde indeholder altså en eller flere elementer fra udfaldsrummet. \\

Ydermere vil et bestemt element i et udfaldsrum eller delmængde denoteres $u_i$ eller $a_i$, altså for henholdsvis $A$ og $U$. Det vil sige hvis vi har en mængde $Q$ vil et bestemt element i $Q$ denoteres som $q_i$. Desuden findes der en række notationer der beskriver delmængders relationer til hinanden og udfaldsrummet.
\begin{enumerate}
  \item $A \cup B$ : udfaldet ligger i enten A eller B, evt. i både A og B
  \item $A \cap B$ : udfaldet ligger i både A og B
  \item $A \backslash B$ : udfaldet ligger i A og ikke B
  \item $A^c$ : udfaldet ligger ikke i A (dette kan også denoteres $\bar{A}$)
\end{enumerate}

\section{Sandsynlighedsfunktion}

Sandsynlighedsfunktion denoteres $P$ og beskriver sandsynligheden for et element i et udfaldsrummet $U$. For eksempel kunne $P(\{1,2\}) = 0.8$ og $P(3) = 0.2$ hvor udfaldsrummet $U = \{1,2,3\}$. Dette betyder at sandsynligheden for $1$ eller $2$ er $80\%$ hvor i mod sandsynligheden for $3$ er $20\%$. Desuden kan vi bruge additions loven til at udlede den totale sandsynlighed for udfaldsrummet \cite{SC}.
\begin{align}
  P(1) + P(2) + P(3) &= 1 \\
  P(\{1,2\}) + P(3) &= 1
\end{align}
Desuden kan $\{1,2\}$ også beskrives som en delmængde af U på formen $A = \{1,2\}$, altså så det desuden er gældende at $P(A) = P(1) + P(2)$

\section{Sandsynlighedsfelt}

Et sandsynlighedsfelt findes på to former, det endelige eller også kaldet det diskrete sandsynlighedsfelt samt det kontinuere sandsynlighedsfelt. Det diskrete sandsynlighedsfelt kan denoteres $(U, P)$ og er bestående af et udfaldsrum $U$ og en sandsynlighedsfunktion $P$. Hvis der er tale om et symmetrisk sandsynlighedsfelt, vil følgende være gældende for $P$.
\begin{align}
  P(U) &= 1 \\
  P(u_1) + P(u_i) + ... + P(u_n) &= 1 \\
  P(u_1) = P(u_i) = ... = P(u_n) &= \frac{1}{|U|}
\end{align}
De to udsagn udtrykker derved at sandsynligheden for $P(U_i)$ er i intervallet $[0;1]$ samt at den samlede sandsynlighed for feltet er $1$, hvor $1 = 100\%$. I et ikke symmetrisk sandsynlighedsfelt vil de to første regler af de overstående tre gælde, men sandsynligheden for de enkle elementer i udfaldsrummet er ikke nødvendigvis lig. \\

Et er endeligt og et er bestående af både et udfaldsrum og dens delmængder samt en sandsynligheds funktion der beskriver sandsynligheden for hvert given  \cite{SC}

\section{Stokastiske variabler}

Stokastiske variabler findes af to type der begge beskrives med $X$, disse to type er henholdsvis diskret og kontinuert. Diskret stokastiske variabler har et endeligt udfaldsrum,

Stokastiske variabler opfører sig på en måde hvor $X$ variables tilstand er alle mulige udfald i udfaldsrummet $U$ i sandsynlighedsfeltet med sandsynlighederne $P(U_n)$

\end{document}
