\documentclass[../../SRP.tex]{subfiles}

\begin{document}

\chapter{Stokastiske variabler}

En væsentlig del af Monte Carlo Metodens teori og praksis bygger på stokastiske variabler. For at forstå stokastiske variabler kræves der en forståelse for simplere begreber indenfor statistik. Disse begreber er: udfaldsrum, delmængde, sandsynlighedsfunktion samt sandsynlighedsfelt. Disse begreber redegøres der for i nedenstående afsnit.

\section{Udfaldsrum og delmængder}

Der findes to typer af udfaldsrum, et diskret og kontinuert udfaldsrum. Forskellen på disse to er, at der i et diskret udfaldsrum kun er et givent antal mulige udfald. Et eksempel på et diskret udfaldsrum er en sekssidet terning. Denne sekssidet terning vil have et udfaldsrum $U$, der indeholder elementerne $1,2,3,4,5,6$. Det kan skrives som $U = \{1,2,3,4,5,6\}$. Altså er der tale om et udfaldsrum med seks elementer, hvilket skrives $|U| = 6$ eller som $n$. Et kontinuert udfaldsrum kunne eksempelvis være højden af elever i en klasse. I dette tilfælde kan udfaldsrummet være defineret som $U = [150cm;210cm]$. Udfaldsrummet er kontinuert da elevers højde kan variere med et infinitesimal. En elev $a$ kan være en uendelig lille stykke højere end elev $b$. Et sådan udfaldsrum indenholder ikke et endeligt antal elementer \cite{SC}. \\

En delmængde er en mængde af et udfaldsrum. Delmængden beskrives ofte med $A$ eller $B$. Hvis vi endnu engang tager udgangspunkt i den sekssidet terning, kan en delmængde af ternings udfaldsrum være $A = \{1,3\}$. En delmængde indeholder derved en eller flere elementer fra udfaldsrummet. \\

Ydermere vil et bestemt element i et udfaldsrum eller delmængde denoteres $u_i$ eller $a_i$, altså for henholdsvis $A$ og $U$. Det vil sige, at hvis vi har en mængde $Q$ vil et bestemt element i $Q$ denoteres som $q_i$. Desuden findes der en række operatorer, der beskriver delmængders relationer til hinanden samt udfaldsrummet.
\begin{enumerate}
  \item $A \cup B$: udfaldet ligger i enten A eller B, evt. i både A og B
  \item $A \cap B$: udfaldet ligger i både A og B
  \item $A \backslash B$: udfaldet ligger i A og ikke B
  \item $A^c$: udfaldet ligger ikke i A (dette kan også denoteres $\bar{A}$)
\end{enumerate}
Betegnelserne er de samme, som vi bruger i mængdelæren, og vi taler derfor om foreningsmængden $A \cup B$, fællesmængden $A \cap B$, mængdedifferensen $A \backslash B$ samt komplementærmængden $A^c$ \cite[p.~16]{SC}. Nedenstående figurer, også kaldet venn-diagrammer, illustrerer overstående mængder og deres operatorer.

\def\firstcircle{(0,0) circle (1.5cm)}
\def\secondcircle{(0:2cm) circle (1.5cm)}

\colorlet{circle edge}{blue!50}
\colorlet{circle area}{blue!20}

\tikzset{filled/.style={fill=circle area, draw=circle edge, thick},
    outline/.style={draw=circle edge, thick}}

\begin{center}
% Set A or B
\begin{tikzpicture}
  \draw[filled] \firstcircle node {$A$}
                \secondcircle node {$B$};
  \node[anchor=south] at (current bounding box.north) {$A \cup B$};
  \draw (-2,-2) rectangle (4,2.2) node [text=black,above] {$U$};
\end{tikzpicture}
\qquad
% Set A and B
\begin{tikzpicture}
    \begin{scope}
        \clip \firstcircle;
        \fill[filled] \secondcircle;
    \end{scope}
    \draw[outline] \firstcircle node {$A$};
    \draw[outline] \secondcircle node {$B$};
    \node[anchor=south] at (current bounding box.north) {$A \cap B$};
    \draw (-2,-2) rectangle (4,2.2) node [text=black,above] {$U$};
\end{tikzpicture}
\end{center}

\begin{center}
%Set not A
\begin{tikzpicture}
  \filldraw[fill=circle area] (-2,-2) rectangle (4,2.2) node [text=black,above] {$U$};
  \filldraw[outline, fill=white] (1,0) circle (1.5cm);
  \node[anchor=south] at (1,1.5) {$\overline{A}$};
\end{tikzpicture}
\qquad
% Set A but not B
\begin{tikzpicture}
  \begin{scope}
    \clip \secondcircle;
    \draw[filled, even odd rule] \firstcircle
                                 \secondcircle node {$B$};
  \end{scope}
  \draw[outline] \firstcircle node {$A$}
                \secondcircle;
  \node[anchor=south] at (current bounding box.north) {$B - A$};
  \draw (-2,-2) rectangle (4,2.2) node [text=black,above] {$U$};
\end{tikzpicture}
\end{center}
\section{Sandsynlighedsfunktion}

Sandsynlighedsfunktion denoteres $P$ og beskriver sandsynligheden for et element i udfaldsrummet $U$. For eksempel kunne $P(\{1,2\}) = 0.8$ og $P(3) = 0.2$, hvor udfaldsrummet $U = \{1,2,3\}$. Dette betyder, at sandsynligheden for $1$ eller $2$ er $80\%$, hvorimod sandsynligheden for $3$ er $20\%$. Desuden kan vi bruge additionsloven til at udlede den totale sandsynlighed for udfaldsrummet \cite{SC}.
\begin{align}
  P(1) + P(2) + P(3) &= 1 \\
  P(\{1,2\}) + P(3) &= 1
\end{align}
Desuden kan $\{1,2\}$ også beskrives som en delmængde af U på formen $A = \{1,2\}$. Herved gælder det, at $P(A) = P(1) + P(2)$. \\

Sandsynlighedsfunktion kan også være afhængig af en anden sandsynlighed. Et eksempel på dette kunne være en kasse is. Kassen med is indeholder fire slags is, hvor $50\%$ af isen er med mørk chokolade og de resterende $50\%$ er med hvid chokolade. Lad denne sandsynlighed værende betegnet med $P_c$. Desuden har $70\%$ af de is der har mørk chokolade, mandler og $50\%$ af isene med hvide chokolade også mandler. Denne vil således være beskrevet som $P(u_i | P_c)$. Herved er sandsynligheden for udfaldet $u_i$ afhængig af $P_c$.

\section{Sandsynlighedsfelt}

Et sandsynlighedsfelt findes på to former. Det endelige - også kaldet det diskrete sandsynlighedsfelt - samt det kontinuærte sandsynlighedsfelt også kaldet et ikke-endeligt sandsynlighedsfelt. Et sandsynlighedsfelt kan denoteres $(U, P)$ og er bestående af et udfaldsrum $U$ og en sandsynlighedsfunktion $P$. Hvis der er tale om et symmetrisk sandsynlighedsfelt, vil følgende være gældende for $P$.
\begin{align}
  P(U) &= 1 \\
  P(u_1) + P(u_i) + \dots + P(u_n) &= 1 \\
  P(u_1) = P(u_i) = \dots = P(u_n) &= \frac{1}{|U|}
\end{align}
De to udsagn udtrykker derved at sandsynligheden for $P(U_i)$ er i intervallet $[0;1]$ samt at den samlede sandsynlighed for feltet er $1$, hvor $1 = 100\%$. I et ikke-symmetrisk sandsynlighedsfelt vil de to første regler af de overstående tre gælde, men sandsynligheden for de enkelte elementer i udfaldsrummet er ikke nødvendigvis lig hinanden. Altså kan $P(u_1) \neq P(u_2)$ \cite{SC}.\\

\section{Stokastiske variabler}

Der findes to typer af stokastiske variabler. Disse er begge denoteres med $X$ og desuden er $x_i$ et konkret udfald. Disse to typer er henholdsvis diskret og kontinuert. En stokastiske variable er en variable, som kan tage alle værdier i et givent udfaldsrum med sandsynligheden $P$, altså $P(X = u_i) = P(u_i)$. Forskellen mellem diskrete og kontinuerte stokastiske variabler er udfaldsrummet og de tilhørende regneregler. Et eksempel på en stokastiske variabel er Europæisk Roulette. Dette spil har et symmetrisk sandsynlighedsfelt med udfaldsrummet $U = \{ 0,1,\dots,36\}$, hvilket indeholder $37 = |U|$ udfald. Det vil altså sige, at den kugle man smider ned i spillet kan beskrives som en stokastiske variabel $X$, og $X$ har en lige stor sandsynlighed for at blive et element i udfaldsrummet, da der er tale om et symmetrisk sandsynlighedsfelt, hvilket betyder at $P(u_1) = P(u_i) = \dots = P(u_n)$, det betyder at $P(X = 1) = P(X = 4)$. Ydermere er roulettens udfaldsrum opdelt i tre farver; grøn, rød og sort. Disse kan betegnes som delmængderne $G$, $R$ og $S$. Vi kan beskrive sandsynligheden for, at den stokastiske variabel $X$ ville være et element i delmængden således $P(X \in S) = P(X \in R) = \frac{18}{37}$ og $P(X \in G) = \frac{1}{37}$. Delmængden $G$ kunne også have været defineret som $G = (R \cup S)^c$, altså de udfald som ikke falder ind under hverken $R$ eller $S$, eller $G = 0$ \cite{NM} \\

Lad desuden $\mathbb{E}[X]$ være den forventede værdi af $X$. Den forventede værdi skal forstås næsten som et gennemsnit. I det ovenstående eksempel vil den forventede værdi $X$ være $\mathbb{E}[X] = 18$. Det kan udregnes på følgende måde:
\begin{equation}
  \mathbb{E}[X] = \sum_{i = 1}^n \frac{u_i}{n}
\end{equation}
Dette er kun tilfældet ved et symmetrisk sandsynlighedsfelt, hvis sandsynligheden for de forskellige udfald er forskellige fra hinanden, altså  $P(u_1) \neq P(U_2)$. Her vil man kunne anvende følgende formel for den forventede værdi af $X$.
\begin{equation}
  \mathbb{E}[X] = \sum_{i = 1}^n u_i \times P(u_i)
\end{equation}

Stokastiske variabler på en computer er ofte repræsenteret som et pseudo-genereret tilfældigt tal, hvilket er et tilfældigt tal udvalgt af en computer. Disse tilfældigt genererede tal bliver brugt til at udføre eksperimenter og simulering af Monte Carlo Metoden \cite{SBM}. Dette vil komme til udtryk i det følgende afsnit hvori det vil blive vist, hvordan Monte Carlo Simulering gør brug af Stokastiske variabler. 

\end{document}
