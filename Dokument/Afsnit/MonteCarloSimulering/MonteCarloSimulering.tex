\documentclass[../../SRP.tex]{subfiles}

\begin{document}

\chapter{Monte Carlo Algoritmer / Simulering}

Monte Carlo Metoden er en række metoder der kan bruges til at læse problemer der afhænger af en række tilfældige tilfælde. Monte Carlo tager udgangs punkt i at nogle problemer er nemmere at  +

Monte Carlo Metoden er en metode inden for statistiske til at løse umiddelbare umulige statistiske problemer. Metoden er empirisk og kan bruges til at lave et udsagn om sandsynelighed om he 

problemer der ikke kan løses teoratisk.

Man kan udføre


Hvis vi tager funktionen $f(x)=$, denne funktion er ikke diffrentierbar da den ikke er kontinuer og kan derfor ikke løses algebragisk. Men vi kan stadig finde dens integral ved brug af Monte Carlo. Vi vælger et tilfældigt punkt, dette punks x værdi bruger vi i funktionen, vi får der efter en y værdi, vi notere os derefter om denne y værdi er højere eller lavere end det tilfældige punkts y værdi. Vi gentage dette tusinds vis af gange til at vi har fået os et forholdshvis stort prøve område. Vi kan nu udregne dets integrale ved at tage arealet det rum vi valgte tilfældige punkter indenfor og derefter gange det med sandsyneligheden for at et punkt er inden for integralet af funktionen, altså sandsyneligheden for at det tilfældige punkt er u

Hvis vi ønsker at lave en numerisk integration ved brug af Monte Carlo Metoden skal vi først finde ud af hvordan vi k



\end{document}
