\documentclass[../../SRP.tex]{subfiles}

\begin{document}

\chapter{Monte Carlo Lokalisering}

Man kan anvende Monte Carlo Metoden til at lokalisere, f.eks. en anatom robot, på et kendt kort. Dette kan lade sig gøre ved brug af Monte Carlo Lokalisering, dette er også en form for partikel filter. Grunden til at Monte Carlo lokalisering er et partikel filter er grundet i at måden hvordan metoden lokalisere. Lad os sige vi har et sandsynlighedsfelt hvor udfaldsrummet $U$ er alle de mulig positioner. Lad $X$ være en stokastisk variable i sandsyneligheds feltet, altså så $X$ er en tilfældig mulig position for robotten på kortet. Vi kan nu lave et eksperiment, hvor $X$ bestemmes til et punkt i sandsynlighedsfeltet, vi kan nu udregne 

Monte Carlo Lokalisering består af et sæt af trin som gentages hver gang den anatom robot bevæger sig. 

Lad os sige at vi har et kvadratisk område med forskelig farvet fliser, som set på nedenstående figur. \\

\Huge{FIGUR!} \\

\section{Opbygning}

\section{implementering}

\end{document}
