\documentclass[../../SRP.tex]{subfiles}

\begin{document}

\chapter{Monte Carlo Lokalisering}
Dette afsnit vil komme ind på hvordan en autonom robot kan implementere en Monte Carlo Algoritme for således at kunne lokalisere sig selv på et kendt kort. Denne Metoder er også kaldet Monte Carlo Lokalisering, og kræver  

Hvis vi tager tager udgangspunkt i en autonom robot, der er har fire hjul og mindst to elektriske motorer med enkoder. Elektriske motorer med enkoder afgiver et elektrisk signal $n$ gange per total rotation af motorrens akse. Således at robotten er i stand til at udregne hvor langt den har. Lad i denne udregning $d$ være diameteren af robottens hjul samt $n$ antallet af impulser per rotation samt $n_t$ værende et antal impulser målt efter en given tid. Længden $L$ som robotten har bevægede sig ved antallet målte impulser $n_t$ kan således udregnes på følgende måde:
\begin{equation}
  L = d \times \pi \times \frac{n_t}{n}
\end{equation}
Vi kan desuden udregne vinklen robotten bevæger sig i, ved at... \\

Desuden har robotten en sensor der kan måle farven på formen $(r,g,b)$ er hvad der befinder sig und

Med udgangspunk i overstående robot der implementer et Dead-reckoning algoritme, kan vi lokalisere robotten på et kendt kort ved brug af Monte Carlo Lokalisering.

Man kan anvende Monte Carlo Metoden til at lokalisere, f.eks. en autonom robot, på et kendt kort. Dette kan lade sig gøre ved brug af Monte Carlo Lokalisering, dette er også en form for partikelfilter. Grunden til at Monte Carlo lokalisering er et partikelfilter er grundet i måden hvorpå metoden lokalisere. Lad os sige vi har et sandsynlighedsfelt hvor udfaldsrummet $U$ er alle de mulig positioner. Lad $X$ være en stokastisk variabel i sandsynligheds feltet, altså så $X$ er en tilfældig mulig position for robotten på kortet. Vi kan nu lave et eksperiment, hvor $X$ bestemmes til et punkt i sandsynlighedsfeltet, vi kan nu udregne 

Monte Carlo Lokalisering består af et sæt af trin som gentages hver gang den autonome robot bevæger sig. 

Lad os sige at vi har et kvadratisk område med forskelig farvet fliser, som set på nedenstående figur. \\

\section{Opbygning}

\begin{enumerate}
  \item Først bestemmes et sæt af stokastisk variabler $X$, som er punkter på formen $(x,y)$ i udfaldsrummet $U$ med højst koncentration omkring de punkter med lavest afvigelse fra målingen $m$

  \item Robotten laver en måling $m$ på formen $(r,g,b)$ på dens nuværende position

  \item Da laves nu en vægtede liste af fejlen mellem $m$ og $rgb(X)$ og punkterne $X$

  \item n antal punkter fjernes
\end{enumerate}

\section{implementering}

\end{document}
